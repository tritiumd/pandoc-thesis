\PassOptionsToPackage{unicode}{hyperref}
\PassOptionsToPackage{hyphens}{url}
\PassOptionsToPackage{dvipsnames,svgnames,x11names,table}{xcolor}
\documentclass[14pt,oneside]{scrbook}

\usepackage{geometry}
\geometry{a4paper,
    top=2.5cm,
    bottom=2.5cm,
    left=3cm,
    right=2.5cm,
}

\usepackage{iftex}

\usepackage[utf8]{vietnam}
\usepackage[utf8]{inputenc}
\usepackage[vietnamese]{babel}

\usepackage{fontspec}
\usepackage{pifont}
\setmainfont{Times New Roman}[
    ]
\setmonofont{JetBrainsMono}[
        Path = /usr/share/fonts/jetbrains-mono/,
        Extension = .ttf,
        UprightFont = *-Regular,
        ItalicFont = *-Italic,
        BoldFont = *-Bold,
        BoldItalicFont = *-BoldItalic,
        Scale = 0.8,
    ]

\usepackage{fontawesome5}
\usepackage{unicode-math}
\usepackage{amsmath}
\usepackage{utfsym}
\usepackage{pmboxdraw}

\usepackage{xcolor}
\definecolor{deepblue}{RGB}{0, 0, 112}
\definecolor{cyclamen}{RGB}{255, 145, 237}
\definecolor{bubblegum}{RGB}{255, 115, 232}
\definecolor{silver}{RGB}{208,208,208}

\usepackage{color}
\usepackage{fancyvrb}
\newcommand{\VerbBar}{|}
\newcommand{\VERB}{\Verb[commandchars=\\\{\}]}
\DefineVerbatimEnvironment{Highlighting}{Verbatim}{commandchars=\\\{\}}
% Add ',fontsize=\small' for more characters per line
\newenvironment{Shaded}{}{}
\newcommand{\AlertTok}[1]{\textcolor[rgb]{1.00,0.33,0.33}{\textbf{#1}}}
\newcommand{\AnnotationTok}[1]{\textcolor[rgb]{0.42,0.45,0.49}{#1}}
\newcommand{\AttributeTok}[1]{\textcolor[rgb]{0.84,0.23,0.29}{#1}}
\newcommand{\BaseNTok}[1]{\textcolor[rgb]{0.00,0.36,0.77}{#1}}
\newcommand{\BuiltInTok}[1]{\textcolor[rgb]{0.84,0.23,0.29}{#1}}
\newcommand{\CharTok}[1]{\textcolor[rgb]{0.01,0.18,0.38}{#1}}
\newcommand{\CommentTok}[1]{\textcolor[rgb]{0.42,0.45,0.49}{#1}}
\newcommand{\CommentVarTok}[1]{\textcolor[rgb]{0.42,0.45,0.49}{#1}}
\newcommand{\ConstantTok}[1]{\textcolor[rgb]{0.00,0.36,0.77}{#1}}
\newcommand{\ControlFlowTok}[1]{\textcolor[rgb]{0.84,0.23,0.29}{#1}}
\newcommand{\DataTypeTok}[1]{\textcolor[rgb]{0.84,0.23,0.29}{#1}}
\newcommand{\DecValTok}[1]{\textcolor[rgb]{0.00,0.36,0.77}{#1}}
\newcommand{\DocumentationTok}[1]{\textcolor[rgb]{0.42,0.45,0.49}{#1}}
\newcommand{\ErrorTok}[1]{\textcolor[rgb]{1.00,0.33,0.33}{\underline{#1}}}
\newcommand{\ExtensionTok}[1]{\textcolor[rgb]{0.84,0.23,0.29}{\textbf{#1}}}
\newcommand{\FloatTok}[1]{\textcolor[rgb]{0.00,0.36,0.77}{#1}}
\newcommand{\FunctionTok}[1]{\textcolor[rgb]{0.44,0.26,0.76}{#1}}
\newcommand{\ImportTok}[1]{\textcolor[rgb]{0.01,0.18,0.38}{#1}}
\newcommand{\InformationTok}[1]{\textcolor[rgb]{0.42,0.45,0.49}{#1}}
\newcommand{\KeywordTok}[1]{\textcolor[rgb]{0.84,0.23,0.29}{#1}}
\newcommand{\NormalTok}[1]{\textcolor[rgb]{0.14,0.16,0.18}{#1}}
\newcommand{\OperatorTok}[1]{\textcolor[rgb]{0.14,0.16,0.18}{#1}}
\newcommand{\OtherTok}[1]{\textcolor[rgb]{0.44,0.26,0.76}{#1}}
\newcommand{\PreprocessorTok}[1]{\textcolor[rgb]{0.84,0.23,0.29}{#1}}
\newcommand{\RegionMarkerTok}[1]{\textcolor[rgb]{0.42,0.45,0.49}{#1}}
\newcommand{\SpecialCharTok}[1]{\textcolor[rgb]{0.00,0.36,0.77}{#1}}
\newcommand{\SpecialStringTok}[1]{\textcolor[rgb]{0.01,0.18,0.38}{#1}}
\newcommand{\StringTok}[1]{\textcolor[rgb]{0.01,0.18,0.38}{#1}}
\newcommand{\VariableTok}[1]{\textcolor[rgb]{0.89,0.38,0.04}{#1}}
\newcommand{\VerbatimStringTok}[1]{\textcolor[rgb]{0.01,0.30,0.58}{#1}}
\newcommand{\WarningTok}[1]{\textcolor[rgb]{1.00,0.33,0.33}{#1}}
\renewcommand{\theFancyVerbLine}{\sffamily \textcolor[rgb]{0.5,0.5,1.0}{\small \oldstylenums{\arabic{FancyVerbLine}}}}
\usepackage{fvextra}
\RecustomVerbatimEnvironment{Highlighting}{Verbatim}{
    numbers=left,
    xleftmargin=1.5em,
    fontsize=\normalsize,
    numberfirstline,
    breaknonspaceingroup=true,
    stepnumber=5,
    breaklines=true,
    breakanywhere=true,
    breakautoindent=true,
    tabsize=2,
    frame=single,
    framesep=3mm,
    rulecolor=\color{silver},
    commandchars=\\\{\}
}\RecustomVerbatimEnvironment{verbatim}{Verbatim}{
    numbers=left,
    xleftmargin=1.5em,
    fontsize=\normalsize,
    numberfirstline,
    breaknonspaceingroup=true,
    stepnumber=5,
    breaklines=true,
    breakanywhere=true,
    breakautoindent=true,
    tabsize=2,
    frame=single,
    framesep=3mm,
    rulecolor=\color{silver},
}

\usepackage[]{hyperref}
\usepackage{bookmark}
\hypersetup{
        citecolor=magenta,
        colorlinks=true,
        linkcolor=black,
        filecolor=bubblegum,
        urlcolor=deepblue,
        pdftitle={Tool tạo pdf báo cáo từ markdown},
}

% from pandoc
\newcommand{\passthrough}[1]{#1}
\usepackage{graphicx}
\makeatletter
\newsavebox\pandoc@box
\newcommand*\pandocbounded[1]{% scales image to fit in text height/width
    \sbox\pandoc@box{#1}%
    \vspace*{-0.75cm}
    \Gscale@div\@tempa{\textheight}{\dimexpr\ht\pandoc@box+\dp\pandoc@box\relax}%
    \Gscale@div\@tempb{\linewidth}{\wd\pandoc@box}%
    \ifdim\@tempb\p@<\@tempa\p@\let\@tempa\@tempb\fi% select the smaller of both
    \ifdim\@tempa\p@<\p@\scalebox{\@tempa}{\usebox\pandoc@box}%
    \else\usebox{\pandoc@box}%
    \fi%
}
\setlength{\abovecaptionskip}{3pt}
\setlength{\belowcaptionskip}{3pt}
% Set default figure placement to htbp
\def\fps@figure{htbp}
\makeatother
\setlength{\emergencystretch}{3em} % prevent overfull lines
\providecommand{\tightlist}{\setlength{\itemsep}{\smallskipamount}\setlength{\parskip}{\smallskipamount}}
\usepackage{soul}
\usepackage{xeCJKfntef}
\renewcommand{\st}[1]{\sout{#1}}
% definitions for citeproc citations
\NewDocumentCommand\citeproctext{}{}
\NewDocumentCommand\citeproc{mm}{%
\begingroup\def\citeproctext{#2}\cite{#1}\endgroup}
\makeatletter
% allow citations to break across lines
\let\@cite@ofmt\@firstofone
% avoid brackets around text for \cite:
\def\@biblabel#1{}
\def\@cite#1#2{{#1\if@tempswa , #2\fi}}
\makeatother
\newlength{\cslhangindent}
\setlength{\cslhangindent}{1.5em}
\newlength{\csllabelwidth}
\setlength{\csllabelwidth}{3em}
\newenvironment{CSLReferences}[2] % #1 hanging-indent, #2 entry-spacing
{\begin{list}{}{%
\setlength{\itemindent}{\smallskipamount}
\setlength{\leftmargin}{0pt}
\setlength{\parsep}{\smallskipamount}
% turn on hanging indent if param 1 is 1
\ifodd #1
\setlength{\leftmargin}{\cslhangindent}
\setlength{\itemindent}{-1\cslhangindent}
\fi
% set entry spacing
\setlength{\itemsep}{#2\baselineskip}}}
{\end{list}}
\newcommand{\CSLBlock}[1]{\hfill\break\parbox[t]{\linewidth}{\strut\ignorespaces#1\strut}}
\newcommand{\CSLLeftMargin}[1]{\parbox[t]{\csllabelwidth}{\strut#1\strut}}
\newcommand{\CSLRightInline}[1]{\parbox[t]{\linewidth - \csllabelwidth}{\strut#1\strut}}
\newcommand{\CSLIndent}[1]{\hspace{\cslhangindent}#1}
% end from pandoc

\usepackage{fancyhdr}
\pagestyle{fancy}
\fancyhf{}
\fancyhead[R]{\small\leftmark} \fancyfoot[C]{\thepage}
\fancypagestyle{plain}{%
\fancyhf{}
\fancyhead[R]{\small\leftmark} \fancyfoot[C]{\thepage}
}

\usepackage{titlesec}
\setcounter{tocdepth}{2}
\renewcommand{\baselinestretch}{1}
\usepackage{indentfirst}
\setlength{\parindent}{14pt}
\setlength{\parskip}{6pt}

\setcounter{secnumdepth}{6}
\renewcommand{\thechapter}{\arabic{chapter}}
\renewcommand{\thesection}{\thechapter.\arabic{section}}
\renewcommand{\thesubsection}{\thesection.\arabic{subsection}}
\renewcommand{\thesubsubsection}{\thesubsection.\arabic{subsubsection}}
\renewcommand{\paragraph}{\arabic{paragraph}}
\renewcommand{\subparagraph}{\alph{subparagraph}}
\titleformat{\chapter}
{\bfseries\fontsize{20}{20}\selectfont}
{\thechapter.}{1em}{}

\titleformat{\section}
{\bfseries\fontsize{18}{20}\selectfont}
{\thesection}{1em}{}

\titleformat{\subsection}
{\bfseries\fontsize{16}{18}\selectfont}
{\thesubsection}{1em}{}

\titleformat{\subsubsection}
{\bfseries\fontsize{16}{16}\selectfont}
{\thesubsubsection}{1em}{}

\titleformat{\paragraph}
{\hspace{1em}\fontsize{14}{16}\selectfont\rmfamily}
{\paragraph}{1em}{}

\titleformat{\subparagraph}
{\hspace{2em}\fontsize{14}{16}\selectfont\rmfamily}
{\subparagraph}{1em}{}

\usepackage{footnotebackref}

%begin tables-vrules.lua
\usepackage{longtable,booktabs,array, multirow}
\usepackage{calc} % for calculating minipage widths
% Correct order of tables after \paragraph or \subparagraph
\usepackage{etoolbox}
\makeatletter
\patchcmd\longtable{\par}{\if@noskipsec\mbox{}\fi\par}{}{}
\makeatother
% Allow footnotes in longtable head/foot
\IfFileExists{footnotehyper.sty}{\usepackage{footnotehyper}}{\usepackage{footnote}}
\makesavenoteenv{longtable}
\setlength{\aboverulesep}{0pt}
\setlength{\belowrulesep}{0pt}
\renewcommand{\arraystretch}{1.3}
%end tables-vrules.lua


\usepackage{multicol}
  \newlength{\currentparskip}
\makeatletter
\@ifpackageloaded{subfig}{}{\usepackage{subfig}}
\@ifpackageloaded{caption}{}{\usepackage{caption}}
\captionsetup[subfloat]{margin=0.5em}
\AtBeginDocument{%
\renewcommand*\figurename{Hình}
\renewcommand*\tablename{Bảng}
}
\AtBeginDocument{%
\renewcommand*\listfigurename{Danh sách hình}
\renewcommand*\listtablename{Danh sách bảng}
}
\newcounter{pandoccrossref@subfigures@footnote@counter}
\newenvironment{pandoccrossrefsubfigures}{%
\setcounter{pandoccrossref@subfigures@footnote@counter}{0}
\begin{figure}\centering%
\gdef\global@pandoccrossref@subfigures@footnotes{}%
\DeclareRobustCommand{\footnote}[1]{\footnotemark%
\stepcounter{pandoccrossref@subfigures@footnote@counter}%
\ifx\global@pandoccrossref@subfigures@footnotes\empty%
\gdef\global@pandoccrossref@subfigures@footnotes{{##1}}%
\else%
\g@addto@macro\global@pandoccrossref@subfigures@footnotes{, {##1}}%
\fi}}%
{\end{figure}%
\addtocounter{footnote}{-\value{pandoccrossref@subfigures@footnote@counter}}
\@for\f:=\global@pandoccrossref@subfigures@footnotes\do{\stepcounter{footnote}\footnotetext{\f}}%
\gdef\global@pandoccrossref@subfigures@footnotes{}}
\@ifpackageloaded{float}{}{\usepackage{float}}
\floatstyle{ruled}
\@ifundefined{c@chapter}{\newfloat{codelisting}{h}{lop}}{\newfloat{codelisting}{h}{lop}[chapter]}
\floatname{codelisting}{Thuật toán}
\newcommand*\listoflistings{\listof{codelisting}{Danh sách thuật toán}}
\makeatother
\usepackage{environ}
\NewEnviron{centerboxed}[1][0.95]{
\makebox[\linewidth][c]{
\centerline{
\begin{minipage}[t][#1\textheight][t]{0.95\linewidth}%
\BODY
\end{minipage}
}
}
}

\NewEnviron{sign}{
\hfill\begin{tabular}{c}
{\it Hà Nội, ngày 5 tháng 11 năm 2023}
\\ \BODY
\\ \\[1cm]
{\bf Phan Thanh Ngọc}\\
\end{tabular}%
}

\newcommand{\toc}[0]{
\tableofcontents
\listoffigures
\listoftables
}
\KOMAoption{listof}{totoc}
\usepackage{tikz}
\newcommand{\centerboxedHeight}{0.95}
\usetikzlibrary{calc}
\newcommand{\tmar}{2.5cm}
\newcommand{\bmar}{2.5cm}
\newcommand{\lmar}{3cm}
\newcommand{\rmar}{2.5cm}
% fix toc use san serif font in header
\addtokomafont{chapterentry}{\rmfamily}
\begin{document}
\frontmatter
\pagestyle{empty}
\begin{titlepage}
        \begin{tikzpicture}[remember picture,overlay]
        \centering
        \draw[blue!70!black,line width=4pt]
        ([xshift=-\rmar+1cm,yshift=-\tmar+0.5cm]current page.north east) coordinate (A)--
        ([xshift=\lmar+0.5cm,yshift=-\tmar+0.5cm]current page.north west) coordinate(B)--
        ([xshift=\lmar+0.5cm,yshift=\bmar-0.5cm]current page.south west) coordinate (C)--
        ([xshift=-\rmar+1cm,yshift=\bmar-0.5cm]current page.south east) coordinate(D)--cycle;
        \draw
        ([xshift=-0.5cm,yshift=0.5cm]A)--
        ([xshift=0.5cm,yshift=0.5cm]B)--
        ([xshift=0.5cm,yshift=-0.5cm]B)--
        ([xshift=-0.5cm,yshift=-0.5cm]B)--
        ([xshift=-0.5cm,yshift=0.5cm]C)--
        ([xshift=0.5cm,yshift=0.5cm]C)--
        ([xshift=0.5cm,yshift=-0.5cm]C)--
        ([xshift=-0.5cm,yshift=-0.5cm]D)--
        ([xshift=-0.5cm,yshift=0.5cm]D)--
        ([xshift=0.5cm,yshift=0.5cm]D)--
        ([xshift=0.5cm,yshift=-0.5cm]A)--
        ([xshift=-0.5cm,yshift=-0.5cm]A)--
        ([xshift=-0.5cm,yshift=0.5cm]A);
        \draw
        ([xshift=0.3cm,yshift=-0.3cm]A)--
        ([xshift=-0.3cm,yshift=-0.3cm]B)--
        ([xshift=-0.3cm,yshift=0.3cm]B)--
        ([xshift=0.3cm,yshift=0.3cm]B)--
        ([xshift=0.3cm,yshift=-0.3cm]C)--
        ([xshift=-0.3cm,yshift=-0.3cm]C)--
        ([xshift=-0.3cm,yshift=0.3cm]C)--
        ([xshift=0.3cm,yshift=0.3cm]D)--
        ([xshift=0.3cm,yshift=-0.3cm]D)--
        ([xshift=-0.3cm,yshift=-0.3cm]D)--
        ([xshift=-0.3cm,yshift=0.3cm]A)--
        ([xshift=0.3cm,yshift=0.3cm]A)--
        ([xshift=0.3cm,yshift=-0.3cm]A);
        \end{tikzpicture}
        \indent
        
\begin{centerboxed}[\centerboxedHeight]
            \centering
            \fontsize{14}{16}\selectfont{
            Đại học Quốc gia Hà Nội \\
            Trường Đại học Khoa học Tự nhiên \\
            }
            \fontsize{13}{16}\selectfont\textbf{Khoa Toán - Cơ - Tin
học} \\
            \fontsize{8}{16}\Pisymbol{dingbat}{69} \hspace{1.4cm}  \Huge\usym{1F56E} \hspace{1cm} \fontsize{8}{16}\Pisymbol{dingbat}{70}\\
            \vspace*{1.5cm}
            \includegraphics[height=3cm,keepaspectratio]{"image/logo.jpg"}\\
            \vspace*{1.5cm}
            \fontsize{18}{1}\selectfont \textbf{Tool tạo pdf báo cáo từ
markdown} \\
            \vspace{2cm}
            \fontsize{14}{14}\selectfont{
             Báo cáo cuối môn học  \\
             Thực tập thực tế phát triển phần mềm  \\
            Ngành: Toán tin  \\
            }
            \vspace{2cm}
            {\fontsize{14}{14}\selectfont\textbf{
            \begin{tabular}{ll}
            Người hướng dẫn:  & Phan Thanh Ngọc
            \end{tabular}
            }}
            \\
            \vspace{2cm}
            \raggedright{\hspace{4.2cm}\fontsize{14}{1}\selectfont \textbf{Người thực hiện:}\\
            \vspace{0.5cm}\hspace{5.5cm}
            \begin{tabular}{ll}
            Phan Thanh Ngọc & 20000001 \\  Phan Thanh
Ngọc & 20000001 \\  
            \end{tabular}
            }
            \vfill
            \centering
            \fontsize{14}{14}\selectfont{\textbf{Hà Nội - 2023}}
            \\\vspace{0.3cm}
        

\end{centerboxed}\end{titlepage}
\begin{titlepage}
        \begin{tikzpicture}[remember picture,overlay]
        \centering
        \draw [line width=2pt]
        ([xshift=\lmar+0.5cm,yshift=-\tmar]current page.north west)
        rectangle
        ([xshift=-\rmar+0.5cm,yshift=\bmar]current page.south east);
        \draw [line width=0.5pt]
        ([xshift=\lmar+0.6cm,yshift=-\tmar-0.1cm]current page.north west)
        rectangle
        ([xshift=-\rmar+0.4cm,yshift=\bmar+0.1cm]current page.south east);
        \end{tikzpicture}
        
\begin{centerboxed}[\centerboxedHeight]
           \centering
           \fontsize{13}{13}\selectfont{
           Đại học Quốc gia Hà Nội \\
           Trường Đại học Khoa học Tự nhiên \\
           \textbf{Khoa Toán - Cơ - Tin học}
           }
           
           \vspace{2.5cm}
           {\fontsize{14}{14}\selectfont \textbf{
           \begin{tabular}{ll}
           Phan Thanh Ngọc \\  Phan Thanh Ngọc \\  
           \end{tabular}
           }}
           \\
           \vspace{2cm}

           {\fontsize{18}{1}\selectfont\textbf{Tool tạo pdf báo cáo từ
markdown} }
           \vfill
           \fontsize{14}{14}\selectfont{
            Báo cáo cuối môn học  \\
            Ngành: Toán tin  \\
           }
           \vfill
           \centering
           \fontsize{14}{14}\selectfont{\textbf{Hà Nội - 2023}}
           \\\vspace{0.3cm}
        

\end{centerboxed}\end{titlepage}
\begin{titlepage}
        \begin{tikzpicture}[remember picture,overlay]
        \centering
        \draw[line width = 3pt]
        ([xshift=\lmar+0.5cm,yshift=-\tmar]current page.north west)
        rectangle
        ([xshift=-\rmar+0.5cm,yshift=\bmar]current page.south east);
        \end{tikzpicture}
        \indent
        
\begin{centerboxed}[\centerboxedHeight]
            \centering
            \fontsize{13}{13}\selectfont{
            Đại học Quốc gia Hà Nội \\
             Trường Đại học Khoa học Tự nhiên \\
            \textbf{Khoa Toán - Cơ - Tin học}
            }
            
            \vspace{2.5cm}
            {\fontsize{14}{14}\selectfont \textbf{
            \begin{tabular}{ll}
            Phan Thanh Ngọc \\  Phan Thanh Ngọc \\  
            \end{tabular}
            }}
            \\
            \vspace{2cm}

            {\fontsize{18}{1}\selectfont \textbf{Tool tạo pdf báo cáo từ
markdown} }
            \vfill
            \fontsize{14}{14}\selectfont{
             Báo cáo cuối môn học  \\
            Ngành: Toán tin \\
            }
            \vfill
            {\fontsize{14}{14}\selectfont \textbf{
            \begin{tabular}{ll}
            Cán bộ hướng dẫn:  & Phan Thanh Ngọc
            \end{tabular}
            }}
            \vfill
            \centering
            \fontsize{14}{14}\selectfont{\textbf{Hà Nội - 2023}}
            \\\vspace{0.3cm}
        

\end{centerboxed}\end{titlepage}
\mainmatter
\pagestyle{fancy}
\setlength{\parindent}{14pt}
\chapter*{Lời mở đầu}\label{lux1eddi-mux1edf-ux111ux1ea7u}
\addcontentsline{toc}{chapter}{Lời mở đầu}

\markboth{Lời mở đầu}{Lời mở đầu}

Tui xin tự cảm ơn tui vì lười gõ latex mà lại chăm ngồi code latex để đẻ
ra con này

\begin{sign}

Sinh viên
\end{sign}

\toc

\listoflistings 

\chapter{Ví dụ cách sử dụng}\label{sec:1}

\section{Đoạn văn}\label{ux111oux1ea1n-vux103n}

Đây là một đoạn văn ví dụ. Trích dẫn theo{[}@texbook{]}: ``Lorem ipsum
dolor sit amet, consectetur adipiscing elit. Maecenas sagittis eleifend
molestie. In fringilla enim dolor. Praesent lacinia dui velit, at
ullamcorper orci laoreet non. Nullam placerat efficitur leo a ornare.
Duis condimentum, tortor vitae pellentesque venenatis, lorem urna
gravida nibh, a fermentum est nunc at magna. Sed posuere arcu odio,
posuere porta libero eleifend vel. Maecenas ut metus quam. Praesent
vitae euismod nisl. Morbi tincidunt nulla eget lacus sagittis aliquet.
Aliquam condimentum commodo lacus, quis convallis quam volutpat id.
Proin lobortis, sapien et fermentum faucibus, lacus est tincidunt augue,
quis facilisis elit nisi non nisl. Donec et odio sit amet ligula laoreet
convallis non quis nulla.''

Đoạn văn thứ 2. Chân trang \footnote{Chân trang}

\section{Danh sách}\label{danh-suxe1ch}

\subsection{Không có thứ tự}\label{khuxf4ng-cuxf3-thux1ee9-tux1ef1}

\begin{itemize}
\tightlist
\item
  1
\item
  2
\item
  3
\end{itemize}

\subsection{Có thứ tự}\label{cuxf3-thux1ee9-tux1ef1}

\begin{enumerate}
\def\labelenumi{\arabic{enumi}.}
\tightlist
\item
  A
\item
  B
\item
  C
\end{enumerate}

Chỉ cần gõ số bắt đầu và sau đó dùng \# để tự cập nhật số

\begin{enumerate}
\def\labelenumi{\alph{enumi})}
\setcounter{enumi}{3}
\tightlist
\item
  D
\item
  E
\item
  F
\end{enumerate}

\subsection{Kết hợp}\label{kux1ebft-hux1ee3p}

\begin{enumerate}
\def\labelenumi{\arabic{enumi}.}
\tightlist
\item
  a
\end{enumerate}

\begin{itemize}
\tightlist
\item
  b
\item
  c

  \begin{itemize}
  \tightlist
  \item
    d
  \item
    e

    \begin{itemize}
    \tightlist
    \item
      f
    \item
      g
    \end{itemize}
  \end{itemize}
\end{itemize}

\begin{enumerate}
\def\labelenumi{\arabic{enumi}.}
\setcounter{enumi}{1}
\tightlist
\item
  h
\end{enumerate}

\section{Bảng biểu}\label{bux1ea3ng-biux1ec3u}

\begin{longtable}[]{@{}
  >{\raggedright\arraybackslash}p{(\linewidth - 6\tabcolsep) * \real{0.3056}}
  >{\raggedright\arraybackslash}p{(\linewidth - 6\tabcolsep) * \real{0.1111}}
  >{\raggedright\arraybackslash}p{(\linewidth - 6\tabcolsep) * \real{0.1111}}
  >{\raggedright\arraybackslash}p{(\linewidth - 6\tabcolsep) * \real{0.1111}}@{}}
\caption{\label{tbl:1}Bảng phức tạp}\tabularnewline
\toprule\noalign{}
\multirow{2}{=}{\begin{minipage}[b]{\linewidth}\raggedright
Location
\end{minipage}} &
\multicolumn{3}{>{\raggedright\arraybackslash}p{(\linewidth - 6\tabcolsep) * \real{0.3333} + 4\tabcolsep}@{}}{%
\begin{minipage}[b]{\linewidth}\raggedright
Temperature 1961-1990 in degree Celsius
\end{minipage}} \\
& \begin{minipage}[b]{\linewidth}\raggedright
min
\end{minipage} & \begin{minipage}[b]{\linewidth}\raggedright
mean
\end{minipage} & \begin{minipage}[b]{\linewidth}\raggedright
max
\end{minipage} \\
\midrule\noalign{}
\endfirsthead
\toprule\noalign{}
\multirow{2}{=}{\begin{minipage}[b]{\linewidth}\raggedright
Location
\end{minipage}} &
\multicolumn{3}{>{\raggedright\arraybackslash}p{(\linewidth - 6\tabcolsep) * \real{0.3333} + 4\tabcolsep}@{}}{%
\begin{minipage}[b]{\linewidth}\raggedright
Temperature 1961-1990 in degree Celsius
\end{minipage}} \\
& \begin{minipage}[b]{\linewidth}\raggedright
min
\end{minipage} & \begin{minipage}[b]{\linewidth}\raggedright
mean
\end{minipage} & \begin{minipage}[b]{\linewidth}\raggedright
max
\end{minipage} \\
\midrule\noalign{}
\endhead
\bottomrule\noalign{}
\endlastfoot
Antarctica & -89.2 & N/A & 19.8 \\
Earth & -89.2 & 14 & 56.7 \\
\end{longtable}

\begin{longtable}[]{@{}rllc@{}}
\caption{Bảng đơn giản}\tabularnewline
\toprule\noalign{}
Right & Left & Default & Center \\
\midrule\noalign{}
\endfirsthead
\toprule\noalign{}
Right & Left & Default & Center \\
\midrule\noalign{}
\endhead
\bottomrule\noalign{}
\endlastfoot
12 & 12 & 12 & 12 \\
123 & 123 & 123 & 123 \\
1 & 1 & 1 & 1 \\
\end{longtable}

\section{Hình ảnh}\label{huxecnh-ux1ea3nh}

\begin{figure}
\centering
\pandocbounded{\includegraphics[keepaspectratio]{image/logo.jpg}}
\caption{Ảnh}\label{fig:1}
\end{figure}

\begin{figure}
\centering
\pandocbounded{\includegraphics[keepaspectratio]{2938c4f05a7d6378358de59bb112bae889715890.png}}
\caption{A simple flowchart.}
\end{figure}

\section{Code}\label{code}

Inline:
\VERB|\BuiltInTok{print}\NormalTok{(}\StringTok{"hello"}\NormalTok{)}|

code block:

\begin{codelisting}\caption{Mergesort}\label{lst:code}\end{codelisting}\vspace*{-0.75cm}

\begin{Shaded}
\begin{Highlighting}[]
\KeywordTok{def}\NormalTok{ merge\_sort(arr):}
    \ControlFlowTok{if} \BuiltInTok{len}\NormalTok{(arr) }\OperatorTok{\textgreater{}} \DecValTok{1}\NormalTok{:}
\NormalTok{        mid }\OperatorTok{=} \BuiltInTok{len}\NormalTok{(arr) }\OperatorTok{//} \DecValTok{2}
\NormalTok{        left\_half }\OperatorTok{=}\NormalTok{ arr[:mid]}
\NormalTok{        right\_half }\OperatorTok{=}\NormalTok{ arr[mid:]}

\NormalTok{        merge\_sort(left\_half)}
\NormalTok{        merge\_sort(right\_half)}

\NormalTok{        i }\OperatorTok{=}\NormalTok{ j }\OperatorTok{=}\NormalTok{ k }\OperatorTok{=} \DecValTok{0}

        \ControlFlowTok{while}\NormalTok{ i }\OperatorTok{\textless{}} \BuiltInTok{len}\NormalTok{(left\_half) }\KeywordTok{and}\NormalTok{ j }\OperatorTok{\textless{}} \BuiltInTok{len}\NormalTok{(right\_half):}
            \ControlFlowTok{if}\NormalTok{ left\_half[i] }\OperatorTok{\textless{}}\NormalTok{ right\_half[j]:}
\NormalTok{                arr[k] }\OperatorTok{=}\NormalTok{ left\_half[i]}
\NormalTok{                i }\OperatorTok{+=} \DecValTok{1}
            \ControlFlowTok{else}\NormalTok{:}
\NormalTok{                arr[k] }\OperatorTok{=}\NormalTok{ right\_half[j]}
\NormalTok{                j }\OperatorTok{+=} \DecValTok{1}
\NormalTok{            k }\OperatorTok{+=} \DecValTok{1}

        \ControlFlowTok{while}\NormalTok{ i }\OperatorTok{\textless{}} \BuiltInTok{len}\NormalTok{(left\_half):}
\NormalTok{            arr[k] }\OperatorTok{=}\NormalTok{ left\_half[i]}
\NormalTok{            i }\OperatorTok{+=} \DecValTok{1}
\NormalTok{            k }\OperatorTok{+=} \DecValTok{1}

        \ControlFlowTok{while}\NormalTok{ j }\OperatorTok{\textless{}} \BuiltInTok{len}\NormalTok{(right\_half):}
\NormalTok{            arr[k] }\OperatorTok{=}\NormalTok{ right\_half[j]}
\NormalTok{            j }\OperatorTok{+=} \DecValTok{1}
\NormalTok{            k }\OperatorTok{+=} \DecValTok{1}

    \ControlFlowTok{return}\NormalTok{ arr}

\CommentTok{\# Example usage:}
\NormalTok{arr }\OperatorTok{=}\NormalTok{ [}\DecValTok{38}\NormalTok{, }\DecValTok{27}\NormalTok{, }\DecValTok{43}\NormalTok{, }\DecValTok{3}\NormalTok{, }\DecValTok{9}\NormalTok{, }\DecValTok{82}\NormalTok{, }\DecValTok{10}\NormalTok{]}
\NormalTok{sorted\_arr }\OperatorTok{=}\NormalTok{ merge\_sort(arr)}
\BuiltInTok{print}\NormalTok{(}\StringTok{"Sorted array is:"}\NormalTok{, sorted\_arr)}
\end{Highlighting}
\end{Shaded}

Thêm từ file

\begin{verbatim}
@book{texbook,
    author = {Cicero},
    year = {45BC},
    title = {De Finibus Bonorum et Malorum},
}
\end{verbatim}

\section{Biểu thức toán học}\label{biux1ec3u-thux1ee9c-touxe1n-hux1ecdc}

Inline: \(a+b\)

Block: \begin{equation}\phantomsection\label{eq:eq1}{a+b}\end{equation}

\section{Liên kết}\label{liuxean-kux1ebft}

Liên kết tới phần~\ref{sec:1}

Liên kết tới bảng~\ref{tbl:1}

Liên kết tới hình~\ref{fig:1}

Liên kết tới biểu thức~\ref{eq:eq1}

Liên kết tới thuật toán~\ref{lst:code}

\chapter{Tài liệu tham khảo}\label{tuxe0i-liux1ec7u-tham-khux1ea3o}

\phantomsection\label{refs}
\begin{CSLReferences}{1}{0}
\bibitem[\citeproctext]{ref-texbook}
1. Cicero. (45BC). \emph{De finibus bonorum et malorum}.

\end{CSLReferences}

\chapter*{Phụ lục}\label{phux1ee5-lux1ee5c}
\addcontentsline{toc}{chapter}{Phụ lục}

\markboth{Phụ lục}{Phụ lục}

Phần này không có số trong mục lục
\backmatter
\end{document}